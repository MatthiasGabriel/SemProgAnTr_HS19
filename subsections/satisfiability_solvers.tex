\subsection{Satisfiability Solvers}
How SAT solvers work and what its limits are.
I only want to describe the most relevant features without going into depth about all the possible optimizations that are currently used.
As a source the paper of the SAT used in STP \cite{10.1007/978-3-540-24605-3_37} could be used. Additionally \cite{Gomes2008SatisfiabilityS} can be used as an alternative

A SAT solver basically tries to find a solution, assignment of all variables so that the formula is truee, to a specified boolean formula, but the formula has to meet certain criteria to be allowed as an input to a SAT solver. The most important criteria is that the formula has to be in the conjunctive normal form, in the further text only called CNF. In the following paragraph the rules of CNF will be explained together with some examples. The next paragraph shows that this restriction to CNF is not a big problem, because it's possible to transform any boolean formula into CNF.\\
In CNF there exist variables like $A$ or $B$ which can either set to true or false.
This variables are the used in so called literals. Each literal consists of exactly one variable or its negation. It's allowed to use a variable in multiple literals, but only one of the values \{true, false\} can be assigned to a variable, which will be used in the whole formula.\\
Example: $A$ or $\lnot B$\\
These literals can be combined to clauses. Each of the clauses consists of any number of literals which are in a disjunctive relation. 
Example: clauseOne = $A \lor B$ clauseTwo = $B \lor \lnot C$ clauseThree = $C$\\
Each formula in CNF consists of any number of clauses which are in an conjuctive relation. Example: $clauseOne \land clauseTwo \land clauseThree$\\
The whole boolean formula would therefore be $A \lor B \land B \lor \lnot C \land C$\\
If a boolean formula is not in the CNF it's neccessary to convert it before entering into the SAT solver. For example the boolean formula $\lnot$(A $\lor$ B) $\lor$ C can be converted into ($\lnot$A $\lor$ C) $\land$ ($\lnot$ B $\lor$ C)\\
There exist different kind of special clauses: empty clause, unit clause, binary clause
The empty clause contains no literal and is in litarature denotes with different symbols. In this publication we use an uppercase lambda $\Lambda$ as suggested by Gomes et al \cite{Gomes2008SatisfiabilityS}
The empty clause always evaluates to false, because there is no possibility to assign a variable which lets the clause evaluate to true.
The unit clause contains exactly one literal. The literal can contain either a variable or its negation. E.g. $A$ $\lnot A$ are both unit clauses. The unit clause is a special case, because we can compute the variable used in the unit clause. If a CNF formula should evaluate to true and contains a unit clause we know that the unit clause must also evaluate to true. Therefore in our example clause  $A$ $A$ has to be true and in the other example $\lnot A$ $A$ has to be false. This property will be heavely used by the SAT solver.
