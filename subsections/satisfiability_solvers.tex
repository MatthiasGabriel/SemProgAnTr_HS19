\subsection{Satisfiability Solvers}
How SAT solvers work and what its limits are.
I only want to describe the most relevant features without going into depth about all the possible optimizations that are currently used.
As a source the paper of the SAT used in STP \cite{10.1007/978-3-540-24605-3_37} could be used. Additionally \cite{Gomes2008SatisfiabilityS} can be used as an alternative\todo{ cleanup and cite properly}

A SAT solver basically tries to find a solution, an assignment of all variables so that the formula is true, to a specified boolean formula. All modern also SAT Solvers return a specific assignment if its possible and otherwise will specify that the formula is not satisfiable.

As an example take the following formula as an input $A \land \lnot B$ there's only one variable assignment where the formula evaluates to true, namely A = true and B = false.

An unsatisfiable example would be the following formula: $A \land \lnot B \land (\lnot A \lor B)$.  There's no possibility how to assign boolean values to A and B which lets the formula evaluate to true as shown by table \ref{table:unsat_truth_table}, therefore its unsatifable.
\begin{table}[!htbp]
\begin{adjustbox}{width=\columnwidth}
\begin{tabular}{ |c|c|c|c|c| } 
 \hline
 $A$ & $B$ & $\lnot B$ & $\lnot A \lor B$ & $A \land \lnot B \land (\lnot A \lor B)$ \\ 
 \hline
 True & True & \highlight{False} & True & False \\ 
 True & False & True & \highlight{False} & False\\ 
 \highlight{False} & True & \highlight{False} & True & False\\ 
 \highlight{False} & False & True & True & False\\ 

 \hline
\end{tabular}
\end{adjustbox}
\caption{Truth table for the unsatisfiable example. The partial expressions that lead to the expression evaluating to false are highlighted.}
\label{table:unsat_truth_table}
\end{table}

To simplify the task, the formula has to meet certain criteria to be allowed as an input to a SAT solver. The most important criteria is that the formula has to be in the conjunctive normal form, in the further text only called CNF. 

In the following paragraph the rules of CNF will be explained together with some examples. The next paragraph shows that this restriction to CNF is not a big problem, because it's possible to transform any boolean formula into CNF\cite{Jackson:2004:CFC:2103144.2103160}.
\subsubsection{CNF}
In CNF there exist variables like $A$ or $B$ which can either set to true or false.
This variables are the used in so called literals. Each literal consists of exactly one variable or its negation. It's allowed to use a variable in multiple literals, but only one of the values \{true, false\} can be assigned to a variable, which will be used in the whole formula.

Consider the two example literals: $A$ and $\lnot B$

These literals can be combined to clauses. Each of the clauses consists of any number of literals which are in a disjunctive relation. 

In our example suppose the following three clauses: 
$$A \lor B$$
$$B \lor \lnot C$$ 
$$C$$
Each formula in CNF consists of any number of clauses which are in an conjuctive relation. For example: $$clauseOne \land clauseTwo \land clauseThree$$\\
The whole boolean formula would therefore be: $$\underbrace{(A \lor B)}_{clauseOne} \land \underbrace{(B \lor \lnot C)}_{clauseTwo} \land \underbrace{(C)}_{clauseThree}$$\\

If a boolean formula is not in the CNF it's neccessary to convert it before entering into the SAT solver. For example the boolean formula $\lnot(A \lor B) \lor C$ can be converted into ($\lnot A \lor C) \land (\lnot B \lor C)$

There exist different kind of special clauses:
\begin{itemize}
\item empty clause
\item unit clause
\item binary clause
\end{itemize}
The empty clause contains no literal and is in litarature denotes with different symbols. In this publication an uppercase lambda $\Lambda$  will be used as suggested by Gomes et al \cite{Gomes2008SatisfiabilityS}.
The empty clause always evaluates to false, because there is no possibility to assign a variable which lets the clause evaluate to true.

The unit clause contains exactly one literal. The literal can contain either a variable or its negation as described before. 
The clauses $A$ and $\lnot A$ are both unit clauses. 
The unit clause is a special case, because its possible to deduce the value of the variable used in the unit clause.
If a CNF formula should evaluate to true and contains a unit clause the unit clause must also evaluate to true. 
This property will be heavely used by the SAT solver.

Consider the example clauses used above. In the clause $A$ the variable $A$ has to be true and in the other example $\lnot A$ the variable $A$ has to be false.
\todo{binary clause}
\subsubsection{DPLL}

There are many different methods how a SAT Formula can be solved. The most simplistic approach would be to just bruteforce the formula and try to evaluate the formula for every possible combination of variables. This works well for a small number of variables but does not scale, because the computation time increases exponentially. In this chapter the the method of the DPLL Procedure will be presented. It is the basis for many of the currently used algorithms and fortunately not to hard to understand.

The predecessor of this method was first presented in 1960 by Davis and Putnam and in 1962 improved by Davis, Logemann and Loveland, hence the name DPLL which consists of the initials of the inventors.

Even though the worst case still requires exponential time, in usual use cases these algorithms provide massive computational benefits. In this publication the recursive version of the algorithm will be used, because its much easier to understand some of the key ideas. Note that the algorithm also exists in iterative versions, which are able to reduce the memory usage.

The algorithm will be represented with 3 parts of pseudo code algorithms \ref{algDPLLRecursive}, \ref{algUnitProgagate} and \ref{algIsUnitClause} of which the first two are essential and the third is added for clarity.

The algorithm DPLLRecursive shows the general structure of the approach, but the algorithm UnitPropagate actually is similiar important.

\begin{algorithm}[caption={DPLLRecursive}, label={algDPLLRecursive}]
 input: $F$: boolean formula in CNF-Form
	$\rho$: assignment of variables
 output: assignment that satisfies F or 
	UNSAT if no assignment is possible
 begin
   $(F, \rho)$  $\gets$ UnitPropagate
   if $\Lambda \in F$
	return UNSAT
   if $F = \emptyset$
	return $\rho$
   $l \gets$ a literal not assigned by $\rho$
   $result \gets$ DPLLRecursive($F \cup \{l\}, \rho$)
   if $result \neq UNSAT$
	return result
   return DPLLRecursive($F \cup \{\lnot l\}, \rho$)
\end{algorithm}

\begin{algorithm}[caption={UnitPropagate}, label={algUnitProgagate}]
 input: $F$: boolean formula in CNF-Form
	$\rho$: assignment of variables
 output: $F$: updated boolean formula
	$\rho$:updated assignment of variables
 begin
   while $\Lambda \notin F$ and $\exists xClause \in F: IsUnitClause(xClause)$
	$\exists! xLiteral \in xClause$
	while $\exists yClause \in F: xLiteral \in yClause$
		$F \gets F \setminus \{yClause\}$     
	while $\exists zClause \in F: \lnot xLiteral \in zClause$
		$newZClause \gets zClause \setminus \{\lnot xLiteral\}$
		$F \gets (F \setminus zClause) \cup newZClause$
	$\rho \gets \rho \cup \{xLiteral\}$
   return $(F, \rho)$
 end
\end{algorithm}
\todo{rearrange UnitPropagate to not overlap}
\todo{maybe rewrite code to be less mathematic}
\begin{algorithm}[caption={IsUnitClause}, label={algIsUnitClause}]
 input: $C$: CNF clause
 output: true if the clause only contains
		exactly one element
	false if the clause contains zero 
		or multiple elements
 begin
   return $|C| = 1$
 end
\end{algorithm}

First the unit propagation is executed, after that its checked if either of the base cases is reached. 
There are only two base cases. Either the formula contains no more clauses which results in a satisfied formula (as shown in lines 9-10) or the formula contains an empty clause which results in UNSAT (as shown in lines 7-8).

The next step(line 11), also called branching step, selects the literal $l$, that should be eliminated in the next recursion.

The last step of the DPLLRecursive is to call itself twice with a new input. In one case a new unit clause with the selected literal $l$ is added to the formula(line 12), in the other case the unit clause with the negation of the literal\footnote{The unit clause containing negation of the literal $l$ is $\lnot l$} is added(line 15).

The step when the first recusion ends in UNSAT and returns to the base is called backtracking and is in the recursive version very simply, because there's literaly nothing to be done. However inn the iterative version, that is used in practise, this the backtracking can be very sophisticated.

The second part of the algorithm, called UnitPropagate, which is called as the first step of the DPLL algoithm, will try to assign values to all variables, which can be deducted.

The easiest case where its possible to determine the value of a variable is if there exists a unit clause. Because CNF defines that every clause has to be true by itself, the clause and even the literal has to be true as well.

Consider the following example: Suppose the unit clause $A$ which is part of the formula, the value true can be assigned to the variable A. On the other hand suppose the unit clause $(\lnot A)$, the value false can be assigned to the variable A.

After a specific value was assigned to the variable the formula can be updated to reflect the changes. 

As a first step all clauses, where the same literal is contained(lines 8-9), have to be removed, as this clauses are evaluated to true. 

The second step is to remove all occurences of the negated literal from all clauses(lines 10-11). This step is possible because the negated literal will always evaluate to false and an $\lor false$ combination is redundant.

This reduction can result in new unit clauses, which then will be propagated, until no more feasable clauses can be found or an empy clause is constructed\footnote{Remember when an empty clause $\Lambda$ is part of a CNF-Formula, the formula always evaluates to false}.

Example for UnitPropagation: 
Assume the formula $F$ is assigned $(A \lor B ) \land (B \lor \lnot C) \land (C)$ and there are currently no variables defined, therefore $\rho$ is empty

\todo{change into graphic}
$UnitPropagate(\{\{(A,B\}, \{B, \lnot C\},\{C\}\}, \{\})$\\
First iteration of while at line 6: \\
$xClause \gets \{C\}$\\
$xLiteral \gets C$\\
First iteration of while at line 8:\\
$yClause \gets \{C\}$\\
$F \gets \{\{(A,B\}, \{B, \lnot C\}\}$\\
First iteration of while at line 10:\\
$zClause \gets \{B, \lnot C\}$ \\
$newZClause \gets \{B\}$\\
$F \gets \{\{(A,B\}, \{B\}\}$ \\
$\rho \gets \{C\}$\\
Second iteration of while at line 6: \\
$xClause \gets \{B\}$\\
$xLiteral \gets B$\\
First iteration of while at line 8:\\
$yClause \gets \{A,B\}$\\
$F \gets \{\{B\}$\\
Second iteration of while at line 8:\\
$yClause \gets \{B\}$\\
$F \gets \{\}$\\
$\rho \gets \{B, C\}$\\
return $\{\}, \{B, C\}$


\subsubsection{Improvements of todays SAT solvers}
The key differences between the original DPLL Algorithm and currently developed SAT solvers lie in the improvement of either the branching selection of the literal, the backtracking after a branch ends in an UNSAT state and some improvements to the unit propagation.
\todo{Some more details}