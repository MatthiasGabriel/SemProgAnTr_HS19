\section{Handwritten vs generated tests}
\todo{Its not possible to test for functionality}
\todo{Its great to test if all possible input is handled correctly}
\todo{Good coverage in short time}
\todo{Generated input can be used to extend the manual test suite}
\todo{write example where a basic handwritten test could've prevented an error which is not detectable by a concolic engine}
\todo{cleanup}
In this section I want show where the generated tests have their (practical) limit and it's advantages/disadvantages in relation to handwritten tests.
I want to show that the generated tests are (only) a complement to the handwritten tests, because they can cover many different paths. But they can only reason about certain programming errors. For example they are useful to detect faulty index access, pointer dereferences or calls of the wrong property in dynamically typed languages. Nevertheless they have no possibility to reason if the output of a program for a specific input is correct or not.