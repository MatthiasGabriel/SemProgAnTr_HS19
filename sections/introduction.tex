\section{Introduction}
In recent years many different publications were published in the field of symbolic and concolic execution. 

The primary purpose of this article is to give an overview of the techniques which are used in symbolic and concolic execution engines.
It mainly focuses on the specific engine KLEE \cite{Cadar:2008:KUA:1855741.1855756}, but many of the presented concepts are found in numerous other applications.
The understanding of these techniques is very important to understand the advantages and disadvantages that arise from using automated testing engines.

The second purpose is to analyse how KLEE behaves in relation to other testing approaches.

Section \ref{section:contraint_solvers} contains some insights about constraints solvers, which are used heavily in symbolic and concolic execution.
Section \ref{section:symbolic_execution} section presents the basics of symbolic execution.
Section \ref{section:concolic_testing} finally focuses on concolic testing based on the example of KLEE \cite{Cadar:2008:KUA:1855741.1855756}. 
Section \ref{section:handwritten_vs_generated_tests} contains a short breakdown of my opinion where concolic testing is helpful and where it's limits are.