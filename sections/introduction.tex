\section{Introduction}
Concolic testing is a method to find programming errors by executing it in a special mode.
Instead of using concrete values the program will be executed on so-called symbolic values.
Because an execution purely based on symbolic values is sometimes not possible, a mixture of concrete and symbolic execution can be performed and will often be referred to as concolic execution.
This article summarizes the basic knowledge that is required to understand how concolic testing works and presents deeper explanations about one specific solution.
Based on concrete examples it's shown which errors can be detected and where manually written tests are still required.


Section \ref{section:contraint_solvers} contains some insights about constraints solvers, which are used heavily in symbolic and concolic execution.
Section \ref{section:symbolic_execution} section presents the basics of symbolic execution.
Section \ref{section:concolic_testing} finally focuses on concolic testing based on the example of KLEE \cite{Cadar:2008:KUA:1855741.1855756}. 
Section \ref{section:handwritten_vs_generated_tests} contains a short breakdown about my opinion where concolic testing is helpful and where it's limits are.
\todo{rewrite introduction}